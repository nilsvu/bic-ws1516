\documentclass[12pt]{article}
 
\usepackage[margin=1in]{geometry} 
\usepackage{amsmath,amsthm,amssymb}
\usepackage{graphicx}
\usepackage{epstopdf}
 
\newcommand{\N}{\mathbb{N}}
\newcommand{\Z}{\mathbb{Z}}
 
\newenvironment{exercise}[2][Exercise]{\begin{trivlist}
\item[\hskip \labelsep {\bfseries #1}\hskip \labelsep {\bfseries #2.}]}{\end{trivlist}}
\newenvironment{problem}[2][Problem]{\begin{trivlist}
\item[\hskip \labelsep {\bfseries #1}\hskip \labelsep {\bfseries #2.}]}{\end{trivlist}}
\usepackage{pdfpages}                    			    % Einbindung von PDF-Seiten
\usepackage{geometry}
\usepackage{courier}
\usepackage{color}
\usepackage{listings}
\definecolor{dkgreen}{rgb}{0,0.6,0}
\definecolor{gray}{rgb}{0.5,0.5,0.5}
\definecolor{ghostwhite}{rgb}{0.97, 0.97, 1.0}
\definecolor{vividviolet}{rgb}{0.62, 0.0, 1.0}

\lstset{language=Matlab,
   keywords={break,case,catch,continue,else,elseif,end,for,function,
      global,if,otherwise,persistent,return,switch,try,while},
   basicstyle=\ttfamily,
   keywordstyle=\color{blue},
   commentstyle=\color{dkgreen},
   stringstyle=\color{vividviolet},
   numbers=left,
   numberstyle=\tiny\color{gray},
   stepnumber=1,
   numbersep=10pt,
   backgroundcolor=\color{ghostwhite},
   tabsize=2,
   showspaces=false,
   showstringspaces=false}

\begin{document}
 
% --------------------------------------------------------------
%                         Start here
% --------------------------------------------------------------
 
\title{Weekly Homework 4}
\author{Benjamin Cramer, Julian G\"oltz\\
Brain Inspired Computing}
 
\maketitle
 
\begin{exercise}{4.1}
Generation of Poisson spike trains \\
\renewcommand{\labelenumi}{\alph{enumi})}
\begin{enumerate}
\item In the lecture we defined the rate of the stochastic process as the ratio of the number of spikes during a time intervall and this intervall $\nu = \frac{N_\textrm{Spikes, T}}{\Delta T}$. If $\langle T\rangle$ is the average ISI, then the number of spikes during a time $T$ is $N_\text{Spikes, T} = \frac{\Delta T}{\langle T \rangle}$, resulting in \[\nu = \frac{1}{\langle T \rangle}.\]
\item blabla
\item
\end{enumerate}

\end{exercise}

\end{document}